\documentclass[task=1]{exercise}
\usepackage{enumitem}
\usepackage{schule}

\newcommand\SJ{Schuljahr 22/23}

\newcommand\makeGlobalHeader[4]{
  \setgroup{#1\\#2}
  \settitle[#4]{#3}
  \addstudent{Datum:}
  \addstudent{~}
}

\newcommand\stufe{Wiederholung}
\newcommand\topic{Mathematik}

\newcommand\makeHeader[2]{
  \makeGlobalHeader{\topic}{\stufe}{#1}{#2}
}


\makeHeader{Impuls\"anderung und Kraft}{Einf\"uhrung}

\renewcommand{\vec}{\overrightarrow}

\begin{document}
  \task[Einwirkung und Kraft]
  Lies im Buch Universum 1 auf Seite 194 den kursiv geschriebenen Absatz und den Paragraphen mit der \"Uberschrift {\bfseries Einwirkung und Kraft}. Vervollst\"andige:
  \begin{enumerate}[label=\textnormal{\alph*)}]
    \item Wenn sich der I \_ \_ \_ \_ \_ eines K\"opers \"a \_ \_ \_ \_ \_, dann \"u\, \_ \_ ein anderer K\"orper eine K \_ \_ \_ \_ auf ihn a \_ \_ .
    \item Was ist das Formelzeichen f\"ur die Kraft und warum ist es genau dieser Buchstabe?\\~\\
  \end{enumerate}
  Erinnere dich nun an die Crash-Tests mit M\"unzen und vervollst\"andige:
  \begin{enumerate}[label=\textnormal{\alph*)}]
  \setcounter{enumi}{2}
    \item Eine M\"unze mit der Masse $m = 5{,}7\,\mathrm{g}$ liegt unbewegt auf dem Tisch. Bevor sie angeschnipst wird, betr\"agt ihr Impuls:\\~\\
    \item Wenn du mit dem Finger gegen sie schnipst und sie anschlie{\ss}end eine Geschwindigkeit von $v = 0{,}75\,\frac{\mathrm{m}}{\mathrm{s}}$ hat, dann ist ihr Impuls nach dem Schnipsen:\\~\\
    \item Erstelle eine Zeichnung wie in Bild 02 auf Seite 195, wo du statt des Fu{\ss}es einen schnipsenden Finger und statt des Balls eine M\"unze einzeichnest. Notiere ebenso wie in Bild 02 die Impulse vor und nach dem Schnipsen mit blauer Schrift und zeichne/schreibe den Kraftpfeil und seine Beschriftung im mittleren Bild mit roter Farbe.
  \end{enumerate}
  
  \newpage
  
  \task[Die Richtung der Kraft]
  Lies im Buch Universum 1 auf Seite 194 -- 195 den Paragraphen mit der \"Uberschrift {\bfseries Die Richtung der Kraft}. Beantworte anschlie{\ss}end folgende Fragen bzw. vervollst\"andige die L\"ucken:
  \begin{enumerate}[label=\textnormal{\alph*)}]
    \item Jede Kraft hat einen B \_ \_ \_ \_ \_ und eine R \_ \_ \_ \_ \_ \_ \_ .\\
    Deshalb ist sie genau wie der Impuls und die Geschwindigkeit\\
    eine v \_ \_ \_ \_ \_ \_ \_ \_ \_ \_ Gr\"o{\ss}e.
    \item Was passiert mit dem Impuls eines K\"orpers, wenn eine Kraft auf ihn einwirkt?\\
    Antwort: Wenn die Kraft auf den K\"orper
    \begin{enumerate}[label=\textnormal{\roman*)}]
      \item in Bewegungsrichtung zeigt, dann\\~\\
      \luecke{\textwidth}.
      \item gegen die Bewegungsrichtung zeigt, dann\\~\\
      \luecke{\textwidth}.
      \item quer zur Bewegungsrichtung zeigt, dann\\~\\
      \luecke{\textwidth}.
    \end{enumerate}
  \end{enumerate}
  
  \task[Woher kommt die Kraft?]
  Lies im Buch Universum 1 auf Seite 195 den Paragraphen mit der \"Uberschrift {\bfseries Woher kommt die Kraft?} und beantworte anschlie{\ss}end folgende Fragen bzw. vervollst\"andige die L\"ucken:
  \begin{enumerate}[label=\textnormal{\alph*)}]
    \item Wenn eine rutschende M\"unze gegen eine Wand prallt, dann \"andert sich ihre\\
    I \_ \_ \_ \_ \_ r \_ \_ \_ \_ \_ \_ \_.
    \item\label{i:reibung} Wenn die M\"unze \"uber den Boden (z.B. Teppich oder Parkett) rutscht, dann wird sie durch R \_ \_ \_ \_ \_ \_ ~~~ l \_ \_ \_ \_ \_ \_ \_ \_ , bis sie schlie{\ss}lich zum Liegen kommt.
    Mikroskopische Erhebungen im Boden, wie z.B. Teppichhaare, \"u \_ \_ \_ eine Kraft auf die M\"unze e \_ \_ \_ \_ \_ \_ \_ g \_ \_ \_ \_ \_ \_ ~~zur Bewegungsrichtung aus.\\~\\
    Der Impulsbetrag wird dadurch \luecke{3cm}.
    \item Mach eine Skizze zu Aufgabe \ref{i:reibung} und kennzeichne wieder die Impulsrichtung mit einem blauen Pfeil und die Kraftrichtung mit einem roten Pfeil.\\\vspace{2cm}\\
    \item Welche Kr\"afte kennst du bereits?\\\vspace{2cm}\\
  \end{enumerate}
  
  \newpage
  
  \task[Kraft und Impuls\"anderung]
  Lies im Buch Universum 1 auf Seite 196 den Paragraphen mit der \"Uberschrift {\bfseries Kraft und Impuls\"anderung} und beantworte anschlie{\ss}end folgende Fragen bzw. vervollst\"andige die L\"ucken:
  \begin{enumerate}[label=\textnormal{\alph*)}]
    \item Je \luecke{3cm} die Kraft ist, die auf einen K\"orper ausge\"ubt wird, desto\\~\\
    \luecke{3cm} ist seine Impuls\"anderung $\Delta p$.\\~\\
    Das $\Delta$ vor dem Formelzeichen $p$ steht hier f\"ur \emph{\"Anderung} oder auch \emph{Differenz}.
    \item\label{i:deltap} Denke nun wieder an die rutschende 20 Cent-M\"unze. Angenommen, sie hat am Anfang einen Impuls von $p = 6\,\mathrm{g}\cdot\frac{\mathrm{m}}{\mathrm{s}}$.\\
    Sie rutscht nun \"uber den Boden, bis sie liegen bleibt und ihr Impuls $0\,\mathrm{g}\cdot\frac{\mathrm{m}}{\mathrm{s}}$ ist.\\
    Wie gro{\ss} ist $\Delta p$?\\
    \item\label{i:deltat} Nimm an, dass die M\"unze aus Aufgabe \ref{i:deltap} genau wie der Ball im Bild 01 auf Seite 196 innerhalb von $\Delta t = 3\,\mathrm{s}$ zum Liegen kommt. Mache eine Skizze wie Bild 01 nur mit M\"unze statt Ball und den Werten aus Aufgabe \ref{i:deltap}.\\\vspace{7cm}\\
  \end{enumerate}
  
  \task[Wie gro{\ss} sind Kr\"afte?]
  Lies im Buch Universum 1 auf Seite 196 den Paragraphen mit der \"Uberschrift {\bfseries Wie gro{\ss} sind Kr\"afte?} und beantworte anschlie{\ss}end folgende Fragen:
  \begin{enumerate}[label=\textnormal{\alph*)}]
    \item Wie lautet die Einheit der Kraft?
    \item Wie gro{\ss} ist die Reibungskraft in Aufgabe 3? Nimm zur Berechnung das $\Delta p$ aus 3\ref{i:deltap} und das $\Delta t$ aus 3\ref{i:deltat} und mache die gleiche Rechnung wie im Buch.
  \end{enumerate}
  \newpage
   
   \task[Heftaufschrieb]
   \"Ubertrage folgenden Text in deinen {\bfseries Physik-Hefter} und f\"ulle dabei die L\"ucken:\\~\\
   \begin{center}
    {\large 1.3 Die Kraft}
   \end{center}
   Um den Impuls eines K\"orpers zu \"andern, muss ein anderer K\"orper auf ihn eine Kraft aus\"uben.\\
   Wir sagen dazu: \glqq Es muss eine Kraft auf den K\"orper wirken.\grqq \\~\\
   Es gilt: \\~\\
   Je \luecke{3cm} die wirkende Kraft, desto \luecke{3cm} die Impuls\"anderung.\\
   Je l\"anger die Kraft wirkt, desto \luecke{3cm} die Impuls\"anderung.\\~\\
   Die Einheit der Kraft ist \luecke{3cm}.\\~\\
   Die Kraft ist eine v \_ \_ \_ \_ \_ \_ \_ \_ \_ \_ Gr\"o{\ss}e.\\
   Das hei{\ss}t, dass sie einen B \_ \_ \_ \_ \_ und eine R \_ \_ \_ \_ \_ \_ \_ hat und durch Pfeile dargestellt werden kann.
\end{document}
