\documentclass[task=1]{exercise}
\usepackage{enumitem}

\newcommand\SJ{Schuljahr 22/23}

\newcommand\makeGlobalHeader[4]{
  \setgroup{#1\\#2}
  \settitle[#4]{#3}
  \addstudent{Datum:}
  \addstudent{~}
}

\newcommand\stufe{Kursstufe}
\newcommand\topic{Elektrostatik}

\newcommand\makeHeader[2]{
  \makeGlobalHeader{\topic}{\stufe}{#1}{#2}
}


\makeHeader{Crash-Tests mit M\"unzen}{Zusammenfassung}

\renewcommand{\vec}{\overrightarrow}

\begin{document}
\paragraph{Hausaufgabe zu heute:} Bildet Gruppen von bis zu drei Leuten und bearbeitet gemeinsam den Crash-Test mit M\"unzen im Buch auf Seite 189. Filmt dabei eure Experimente und die Beantwortung der Fragen und ladet eure Filme auf Moodle hoch.

Auf diesem Arbeitsblatt wollen wir eure Ergebnisse wiederholen und zusammenfassen.

  \task[Beobachtung zu V1]
  Wenn die angesto{\ss}ene 20 Cent-M\"unze gegen die liegende prallt, dann \_\_\_\_\_\_\_\_ die erste 20 Cent-M\"unze \_\_\_\_\_\_\_\_, w\"ahrend die zweite \_\_\_\_\_\_\_\_\_\_\_\_\_\_\_\_\_\_\_\_.\\
  Je st\"arker die erste M\"unze angesto{\ss}en wurde, desto \_\_\_\_\_\_\_\_\_\_ rutscht die zweite.
  
  \task[Beobachtung zu V2]
  
  \begin{enumerate}[label=\textnormal{\roman*)}]
   \item Was macht die schwere M\"unze nach dem Sto{\ss} in V2a)?\\\vspace{1cm}\\
   Was macht die 20 Cent-M\"unze nach dem Sto{\ss}?\\\vspace{1cm}\\
   \item Was macht die leichte M\"unze nach dem Sto{\ss} in V2b)?\\\vspace{1cm}\\
   Was macht die 20 Cent-M\"unze nach dem Sto{\ss}?\\\vspace{1cm}\\
   \item\label{item:s} Versuche, die Ergebnisse aus i) und ii) zusammenzufassen:\\
   Je schwerer die erste M\"unze, desto \_\_\_\_\_\_\_\_\_\_ rutscht die 20 Cent-M\"unze.\\
   Je leichter die erste M\"unze, desto \_\_\_\_\_\_\_\_\_\_ bleibt die 20 Cent-M\"unze liegen.
   \item Benutze nun deine Erkenntnisse aus V1 und V2a) und b), um eine Strategie f\"ur V2c) zu entwickeln.\\
   \vspace{3cm}
   \end{enumerate}
   
   \task[Weitere \"Uberlegungen]
  \begin{enumerate}[label=\textnormal{\roman*)}]
   \item Warum bleibt die 20 Cent-M\"unze \"uberhaupt liegen und gleitet nicht immer weiter \"uber die Unterlage?\\
   \vspace{3cm}
   \item Liste hier ein paar Faktoren auf, die die L\"ange des Bremsweges der M\"unze beeinflussen k\"onnen.\\
   \vspace{3cm}
   \item\label{item:vs} Vervollst\"andige nun speziell diesen Satz:\\
   Je \_\_\_\_\_\_\_\_\_\_\_\_ die Geschwindigkeit der 20 Cent-M\"unze nach dem Sto{\ss}, desto \_\_\_\_\_\_\_\_\_\_\_\_ ihr Bremsweg.
   \item Benutze nun die Ergebnisse aus \ref{item:vs} und Aufgabe 2 \ref{item:s} um folgende S\"atze zu vervollst\"andigen:\\
   Je schwerer die erste M\"unze, desto \_\_\_\_\_\_\_\_ die Geschwindigkeit der 20 Cent-M\"unze direkt nach dem Sto{\ss}.\\
   Je leichter die erste M\"unze, desto \_\_\_\_\_\_\_\_ die Geschwindigkeit der 20 Cent-M\"unze direkt nach dem Sto{\ss}.
   
  \end{enumerate}
  
  
\end{document}
