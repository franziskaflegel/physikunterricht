\documentclass[task=1]{exercise}
\usepackage{enumitem}

\newcommand\SJ{Schuljahr 22/23}

\newcommand\makeGlobalHeader[4]{
  \setgroup{#1\\#2}
  \settitle[#4]{#3}
  \addstudent{Datum:}
  \addstudent{~}
}

\newcommand\stufe{Kursstufe}
\newcommand\topic{Elektrostatik}

\newcommand\makeHeader[2]{
  \makeGlobalHeader{\topic}{\stufe}{#1}{#2}
}


\makeHeader{Impuls}{Wiederholung}

\renewcommand{\vec}{\overrightarrow}

\begin{document}
  \task[Einheiten]  
  \begin{enumerate}[label=\textnormal{\alph*)}]
   \item Nenne drei m\"ogliche Einheiten f\"ur die Masse $m$:\\\vspace{.1cm}\\
   \item Nenne drei m\"ogliche Einheiten f\"ur die Geschwindigkeit $v$:\\\vspace{.1cm}\\
   \item Nenne drei m\"ogliche Einheiten f\"ur den Impuls $p$:\\\vspace{.1cm}\\
   \end{enumerate}
   
   \task[Impuls-Berechnung]
   \begin{enumerate}[label=\textnormal{\alph*)}]
     \item Berechne die Impulse folgender Objekte:
     \begin{enumerate}[label=\textnormal{\roman*)}]
       \item ein rennendes Kind ($m = 30\,\mathrm{kg}$, $v = 5\,\frac{\mathrm{m}}{\mathrm{s}}$)\\\vspace{.3cm}\\
       \item ein Wanderfalke im Sturzflug ($m = 1\,\mathrm{kg}$, $v = 300\,\frac{\mathrm{km}}{\mathrm{h}}$)\\\vspace{.3cm}\\
       \item ein Kreuzfahrtschiff in voller Fahrt ($m = 70\,000\,\mathrm{t}$, $v = 50\,\frac{\mathrm{km}}{\mathrm{h}}$)\\\vspace{.3cm}\\
       \item eine abgeschossene Pistolenkugel ($m = 2\,\mathrm{g}$, $v = 300\,\frac{\mathrm{m}}{\mathrm{s}}$)\\\vspace{.3cm}\\
     \end{enumerate}
     \item Ordne die Impulse aus a) der Gr\"o{\ss}e nach.\\\vspace{.3cm}\\
   \end{enumerate}
   
   \task[Impulserhaltung]
   \begin{enumerate}[label=\textnormal{\alph*)}]
    \item Ein Spielzeugwagen mit der Masse $m = 100\,\mathrm{g}$ bewegt sich mit einer Geschwindigkeit von $v = 0{,}12\,\frac{\mathrm{m}}{\mathrm{s}}$. Wie gro{\ss} ist sein Impuls?\\\vspace{.3cm}\\
    \item Zwei Spielzeugw\"agen, die zusammenh\"angen und \emph{jeweils} eine Masse $m = 100\,\mathrm{g}$ haben, sollen \emph{insgesamt} den gleichen Impuls haben wie im Ergebnis von a). Wie gro{\ss} ist dann ihre Geschwindigkeit?
   \end{enumerate}
\end{document}
