\documentclass[task=1]{exercise}
\usepackage{enumitem}
\usepackage{schule}
\usepackage{harpoon}

\newcommand\SJ{Schuljahr 22/23}

\newcommand\makeGlobalHeader[4]{
  \setgroup{#1\\#2}
  \settitle[#4]{#3}
  \addstudent{Datum:}
  \addstudent{~}
}

\newcommand\stufe{Kursstufe}
\newcommand\topic{Elektrostatik}

\newcommand\makeHeader[2]{
  \makeGlobalHeader{\topic}{\stufe}{#1}{#2}
}


\makeHeader{Naturphänomene Elektrostatik}{Elektrisches Feld der Erde}

\renewcommand{\vec}{\overrightharp}

\begin{document}
  \task[Feldlinienbild]
  Die Erde ist gegenüber der sie umgebenden Luftschicht insgesamt negativ geladen.
  Wenn wir vereinfacht davon ausgingen, dass die Erde eine geladene Kugel im Vakuum wäre,
  wie sähe dann das durch sie hervorgerufene elektrische Feld aus?\\
  Rufe dir zuerst die Regeln zum Zeichnen der Feldlinien in Erinnerung und vervollständige die Lücken.\\
  \begin{itemize}
   \item In einem statischen elektrischen Feld (ohne Strom und ohne Magnete) beginnen elektrische Feldlinien beginnen immer beim \luecke{3cm} Pol und enden beim \luecke{3cm} Pol oder im \luecke{3cm}. (Ausnahme: Sie beginnen oder enden an elektrischen Leitern. Sie beginnen und enden aber niemals einfach so im Raum und sind nicht in sich geschlossen.)
   \item In der Elektrostatik treten Feldlinien \luecke{3cm} aus elektrischen Leitern aus.
   \item Feldlinien \luecke{3cm} sich nicht. (Warum nochmal?)
   \item Die Coulomb-Kraft auf einen positiv geladenen Probe-Körper in einem elektrischen Feld wirkt \luecke{3cm} zu den gezeichneten Feldlinien.
   \item Die Feldliniendichte ist proportional zum Betrag der \luecke{3cm} und damit auch zum Betrag der \luecke{5cm}.
  \end{itemize}

  
  Zeichne nun das Feldlinienbild für das durch die Erde hervorgerufene elektrische Feld, wenn wir annehmen, dass sie eine geladene Kugel im Vakuum wäre.\\
  \vspace{4cm}~\\
  \task[Proportionalitätsgesetz]
  Wenn die Erde einfach eine geladene Kugel im Vakuum wäre, wie (d.h. mit welchem Proportionalitätsgesetz) würde der Betrag $E(r)$ des elektrischen Feldes mit dem Abstand $r$ zum Mittelpunkt der Erde abnehmen?\\
  \vspace{1cm}~\\
  \newpage
  \task[Tatsächlicher Verlauf des elektrischen Feldes]
  \begin{enumerate}
  \item In Wirklichkeit fällt der Betrag $E(r)$ des elektrischen Feldes in Abhängigkeit von $r$ jedoch viel schneller ab, als durch Aufgabe 2 vorhergesagt. Fülle die mittleren zwei Spalten in unten stehender Tabelle aus und vergleiche die letzten beiden Spalten.\\~\\~\\
  \begin{tabular}{ccccc}
    \shortstack{Abstand $h$ zur\\Erdoberfläche} & \shortstack{Abstand $r$ zum\\Erdmittelpunkt} & \shortstack{Betrag des\\elektrischen Feldes $E(r)$\\entsprechend b)} & \shortstack{tatsächlich gemessener\\Betrag des\\elektrischen Feldes}\\\hline~\\
    0\,m   & $637 \cdot 10^4$\,m & 130\,$\frac{\mathrm{V}}{\mathrm{m}}$ & 130\,$\frac{\mathrm{V}}{\mathrm{m}}$\\~\\
    $10^4$\,m  & & & 13\,$\frac{\mathrm{V}}{\mathrm{m}}$\\~\\
    $2\cdot 10^4$\,m  & & & 1,3\,$\frac{\mathrm{V}}{\mathrm{m}}$\\~\\\hline
  \end{tabular}~\\~\\~\\
  Platz zum Rechnen:\\\vspace{5cm}\\
  \item Kannst du eine Erklärung dafür finden, dass das elektrische Feld der Erde in Wirklichkeit schneller mit der Höhe $h$ abnimmt als durch Aufgabe 2 vorhergesagt?\\
  Überlege dir dafür, was mit den Feldlinien passieren muss. Sie dürfen ja nicht im Nichts beginnen oder enden. Und trotzdem müssen sie nach außen hin weniger werden.
  \end{enumerate}
  \end{document}
