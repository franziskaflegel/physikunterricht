\documentclass[task=1]{exercise}
\usepackage{enumitem}
\usepackage{schule}
\usepackage{harpoon}

\newcommand\SJ{Schuljahr 22/23}

\newcommand\makeGlobalHeader[4]{
  \setgroup{#1\\#2}
  \settitle[#4]{#3}
  \addstudent{Datum:}
  \addstudent{~}
}

\newcommand\stufe{Kursstufe}
\newcommand\topic{Elektrostatik}

\newcommand\makeHeader[2]{
  \makeGlobalHeader{\topic}{\stufe}{#1}{#2}
}


\makeHeader{Coulomb-Kraft und elektrisches Feld}{Wiederholung Teil 2}

\renewcommand{\vec}{\overrightharp}

\begin{document}
  \task[Coulomb-Kraft]
  \begin{enumerate}[label=\textnormal{\alph*)}]
    \item Sei $q = 10^{-6}\,\mathrm{C}$ und $Q = 2\cdot 10^{-8}\,\mathrm{C}$ und $r = 2\cdot 10^{-9}\,\mathrm{m}$. Berechne den Betrag der Coulomb-Kraft $F_\mathrm{C}$.
    \begin{align*}
      \left| F_\mathrm{C} \right| \, &=
    \end{align*}
  \end{enumerate}  
  
  \task[Elektrisches Feld]
  \begin{enumerate}[label=\textnormal{\alph*)}]
    \item Eine positive Ladung $q$ befindet sich in einem elektrischen Feld $\vec{E}$. In welche Richtung zeigt die Coulomb-Kraft $\vec{F}_\mathrm{C}$, die auf $q$ wirkt?\\~\\~\\
    In welche Richtung w\"urde sie zeigen, wenn $q$ eine negative Ladung w\"are?\\~\\
    \item Bei einem Gewitter kurz vor einem Blitz hat das elektrische Feld eine ungef\"ahre Feldst\"arke von $3 \cdot 10^6 \frac{\mathrm{N}}{\mathrm{C}}$. Berechne den Betrag der Coulomb-Kraft $F_\mathrm{C}$, die ein negativ geladender Wassertropfen mit einer Ladung von $q = -10\,\mathrm{nC}$ erf\"ahrt.
    \begin{align*}
      \left| F_\mathrm{C} \right| \, &=
    \end{align*}
    \item Was passiert dabei mit den Wassermolekülen innerhalb des Wassertropfens und wie nennt man dieses Phänomen?\\~\\
  \end{enumerate}
  
  \task[Ph\"anomene]
  \begin{enumerate}[label=\textnormal{\alph*)}]
    \item Verbinde korrekt.\\~\\~\\
    \begin{tabular}{p{0.2\textwidth}p{0.2\textwidth}l}
      Influenz    & Isolator  & geringfügige Verschiebung von Elektronen\\~\\
      Polarisation& Leiter    & Verschiebung von frei beweglichen Elektronen\\
    \end{tabular}~\\~\\
    \item Um welches der beiden Phänomene aus a) handelt es sich, wenn ein geladener Gegenstand in die Nähe folgender Materialien gebracht wird?\\
    \begin{itemize}
     \item Aluminium
     \item Glas
     \item Porzellan
     \item Eisen
     \item Papier
     \item Kupfer
     \item Grieß
    \end{itemize}

  \end{enumerate}
  \end{document}
