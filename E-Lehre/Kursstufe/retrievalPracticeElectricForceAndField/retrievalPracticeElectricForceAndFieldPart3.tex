\documentclass[task=1]{exercise}
\usepackage{enumitem}
\usepackage{schule}
\usepackage{harpoon}

\newcommand\SJ{Schuljahr 22/23}

\newcommand\makeGlobalHeader[4]{
  \setgroup{#1\\#2}
  \settitle[#4]{#3}
  \addstudent{Datum:}
  \addstudent{~}
}

\newcommand\stufe{Wiederholung}
\newcommand\topic{Mathematik}

\newcommand\makeHeader[2]{
  \makeGlobalHeader{\topic}{\stufe}{#1}{#2}
}


\makeHeader{Coulomb-Kraft und elektrisches Feld}{Wiederholung Teil 3}

\renewcommand{\vec}{\overrightharp}

\begin{document}
  \task[Geladener Stab an Wasserstrahl]
  Ein geladener Stab wird in die Nähe eines dünnes Wasserstrahls gehalten.
  \begin{enumerate}[label=\textnormal{\alph*)}]
    \item Beschreibe die zu erwartende Beobachtung, ohne diese zu erklären.\\~\\~\\
    \item Erkläre nun die zu erwartende Beobachtung. Nimm ggf.\ eine Skizze zur Hilfe.\\~\\~\\~\\
    \item Der Fachbegriff für dieses Phänomen ist:\\~\\
    O \_ \_ \_ \_ t \_ \_ \_ \_ \_ \_ \_ p \_ \_ \_ \_ \_ \_ \_ \_ \_ \_ \_ .
  \end{enumerate}  
  
  \task[Luftballon an Zimmerdecke]
  Reibt man einen Luftballon an einem Wollpullover und berührt damit die Zimmerdecke, so haftet er daran.
  \begin{enumerate}[label=\textnormal{\alph*)}]
    \item Erkläre diese Beobachtung und nimm ggf.\ eine Skizze zur Hilfe.\\~\\~\\~\\
    \item Der Fachbegriff für dieses Phänomen ist:\\~\\
    V \_ \_ s \_ \_ \_ \_ \_ \_ \_ \_ \_ p \_ \_ \_ \_ \_ \_ \_ \_ \_ \_ \_ .
  \end{enumerate} 
  
  \task[FFP2 - Maske]
  Wie filtert eine FFP2-Maske die unerwünschten Partikel aus der Atemluft? Benutze dabei das Wort \emph{Polarisation}.\\~\\~\\
  
  \task[Leiter unter Gewitterwolke]
  Während eines Gewitters befindet sich über uns eine Wolke, deren Unterseite stark negativ geladen ist. Welchen Effekt hat die negative Ladung auf elektrische Leiter, wie z.B. Menschen oder die Erde?\\~\\~\\~\\~\\
  \end{document}
