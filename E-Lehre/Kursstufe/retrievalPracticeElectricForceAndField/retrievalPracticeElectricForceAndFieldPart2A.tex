\documentclass[task=1]{exercise}
\usepackage{enumitem}
\usepackage{schule}
\usepackage{harpoon}

\newcommand\SJ{Schuljahr 22/23}

\newcommand\makeGlobalHeader[4]{
  \setgroup{#1\\#2}
  \settitle[#4]{#3}
  \addstudent{Datum:}
  \addstudent{~}
}

\newcommand\stufe{Wiederholung}
\newcommand\topic{Mathematik}

\newcommand\makeHeader[2]{
  \makeGlobalHeader{\topic}{\stufe}{#1}{#2}
}


\makeHeader{Coulomb-Kraft und elektrisches Feld}{Wiederholung Teil 2A}

\renewcommand{\vec}{\overrightharp}

\begin{document}
  \task[Coulomb-Kraft]
  \begin{enumerate}[label=\textnormal{\alph*)}]
    \item Gegeben seien zwei Ladungen $q$ und $Q$ im Abstand $r$. Wie lautet die Formel f\"ur die den Betrag der Coulomb-Kraft $F_\mathrm{C}$, die zwischen $Q$ und $q$ wirkt?
    \begin{align*}
      \left| F_\mathrm{C} \right| \, &=
    \end{align*}
    (Tipp: Wie lauten die Proportionalit\"aten zwischen der Coulomb-Kraft $F_\mathrm{C}$ und den Gr\"o{\ss}en $q$, $Q$ und $r$? Je gr\"o{\ss}er $q$, desto \ldots)
    \item Was ist die Einheit der Coulomb-{\bfseries Kraft} $F_\mathrm{C}$?\\~\\
    \item\label{i:calcFC} Sei $q = 3\,\mathrm{C}$ und $Q = 4\,\mathrm{C}$ und $r = 3\,\mathrm{m}$. Berechne den Betrag der Coulomb-Kraft $F_\mathrm{C}$. Nimm dazu an, dass $k\approx 9\cdot 10^9 \frac{\mathrm{Vm}}{\mathrm{C}}$.\\~\\
    geg.:\hspace{7cm} ges.:~~~$\left| F_\mathrm{C} \right|$\\\vspace{1cm}
    \item\label{i:units} Führe eine kurze Einheitenüberprüfung für Aufgabe \ref{i:calcFC} durch. Du darfst dazu annehmen, dass du schon weißt, dass $\frac{\mathrm{V}}{\mathrm{m}} = \frac{\mathrm{N}}{\mathrm{C}}$ ist.\\
    (Hilfestellung: Was ist nochmal die Einheit für Kraft? Was ist die Einheit von Ladung?)\\\vspace{.5cm}
    \item Wovon ist $\frac{\mathrm{V}}{\mathrm{m}} = \frac{\mathrm{N}}{\mathrm{C}}$ (was wir gerade in Aufgabe \ref{i:units} gebraucht haben) nochmal die Einheit?\\\vspace{.5cm}
    \item Sei $q = 3\cdot 10^{-6}\,\mathrm{C}$ und $Q = 4\cdot 10^{-9}\,\mathrm{C}$ und $r = 3\cdot 10^{-7}\,\mathrm{m}$. Berechne den Betrag der Coulomb-Kraft $F_\mathrm{C}$. Nimm dazu an, dass $k\approx 9\cdot 10^9 \frac{\mathrm{Vm}}{\mathrm{C}}$.\\~\\
    geg.:\hspace{7cm} ges.:~~~$\left| F_\mathrm{C} \right|$\\\vspace{.5cm}
  \end{enumerate}  
  
  \task[Elektrisches Feld]
  \begin{enumerate}[label=\textnormal{\alph*)}]
    \item Eine positive Ladung $q$ befindet sich an einem Ort, wo der Betrag der elektrischen Feldst\"arke gleich $E$ ist. Wie lautet die Formel f\"ur den Betrag der Coulomb-Kraft $F_\mathrm{C}$, die auf die Ladung $q$ wirkt?
    \begin{align*}
      \left| F_\mathrm{C} \right| \, &=
    \end{align*}
    \item Bei einem Gewitter kurz vor einem Blitz hat das elektrische Feld eine ungef\"ahre Feldst\"arke von $3{,}2 \cdot 10^6 \frac{\mathrm{N}}{\mathrm{C}}$. Berechne den Betrag der Coulomb-Kraft $F_\mathrm{C}$, die ein negativ geladender Wassertropfen mit einer Ladung von $q = -20\,\mathrm{nC}$ erf\"ahrt.\\
    (Achtung! $1\,\mathrm{nC} = 10^{-9}\,\mathrm{C}$)\\~\\
    geg.:\hspace{7cm} ges.:~~~$\left| F_\mathrm{C} \right|$\\\vspace{1cm}
  \end{enumerate}

  \end{document}
