\documentclass[task=1]{exercise}
\usepackage{enumitem}
\usepackage{schule}
\usepackage{harpoon}

\newcommand\SJ{Schuljahr 22/23}

\newcommand\makeGlobalHeader[4]{
  \setgroup{#1\\#2}
  \settitle[#4]{#3}
  \addstudent{Datum:}
  \addstudent{~}
}

\newcommand\stufe{Wiederholung}
\newcommand\topic{Mathematik}

\newcommand\makeHeader[2]{
  \makeGlobalHeader{\topic}{\stufe}{#1}{#2}
}


\makeHeader{Coulomb-Kraft und elektrisches Feld}{Wiederholung Teil 1}

\renewcommand{\vec}{\overrightharp}

\begin{document}
  \task[Coulomb-Kraft]
  \begin{enumerate}[label=\textnormal{\alph*)}]
    \item Gegeben seien zwei Ladungen $q$ und $Q$ im Abstand $r$. Wie lautet die Formel f\"ur die den Betrag der Coulomb-Kraft $F_\mathrm{C}$, die zwischen $Q$ und $q$ wirkt?
    \begin{align*}
      \left| F_\mathrm{C} \right| \, &=
    \end{align*}
    (Tipp: Wie lauten die Proportionalit\"aten zwischen der Coulomb-Kraft $F_\mathrm{C}$ und den Gr\"o{\ss}en $q$, $Q$ und $r$? Je gr\"o{\ss}er $q$, desto \ldots)
    \item Sei $q = 1\,\mathrm{C}$ und $Q = 2\,\mathrm{C}$ und $r = 2\,\mathrm{m}$. Berechne den Betrag der Coulomb-Kraft $F_\mathrm{C}$.
    \begin{align*}
      \left| F_\mathrm{C} \right| \, &=
    \end{align*}
  \end{enumerate}
  
  \task[Elektrisches Feld]
  \begin{enumerate}[label=\textnormal{\alph*)}]
    \item Eine positive Ladung $q$ befindet sich an einem Ort, wo der Betrag der elektrischen Feldst\"arke gleich $E$ ist. Wie lautet die Formel f\"ur den Betrag der Coulomb-Kraft $F_\mathrm{C}$, die auf die Ladung $q$ wirkt?
    \begin{align*}
      \left| F_\mathrm{C} \right| \, &=
    \end{align*}
    \item Nenne eine Einheit f\"ur das elektrische Feld.\\
    \item Vervollst\"andige:\\
    Das elektrische Feld ordnet jedem Punkt im Raum einen B \_ \_ \_ \_ \_ und eine\\
    R \_ \_ \_ \_ \_ \_ \_ zu. Deshalb ist es ein V \_ \_ \_ \_ \_ feld.
  \end{enumerate}
  
  \task[Ph\"anomene]
    \begin{minipage}{.45\linewidth}
    Wie nennt man die Verschiebung von frei beweglichen Elektronen unabhängig von ihren zugehörigen Atomkernen in einem elektrischen Leiter?\\~\\
     I \_ \_ \_ \_ \_ \_ \_
    \end{minipage}\hfill
    \begin{minipage}{.4\linewidth}
     \includegraphics[height=5cm]{images/influenz.jpg}
    \end{minipage}
    \begin{minipage}{.45\linewidth}
    Wie nennt man die geringfügige Verschiebung von Elektronen von ihrem Atomkern in einem elektrischen Isolator?\\~\\
        P \_ \_ \_ \_ \_ \_ \_ \_ \_ \_ \_
    \end{minipage}\hfill
    \begin{minipage}{.4\linewidth}
     \includegraphics[height=4cm]{images/polarisation.jpg}
    \end{minipage}  
  \end{document}
